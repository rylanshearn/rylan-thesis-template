
\fussy % more stringent threshold for gaps between words
\thispagestyle{empty}
\cleartooddpage

\noindent
\lipsum % dummy text fill

\section{Figure Reference Example}
The importance of sexual reproduction has been contemplated since the beginnings of evolutionary theory \parencite{darwin_effects_1876}. One of the earliest theories suggested that sexual reproduction exists because it provides variation \parencite{weismann_significance_1889,weismann_amphimixis_1892}, which was later stated in the context of population genetics as providing accelerated evolution (\cref{fig:introfigure1}; \cite{,fisher_genetical_1930,muller_genetic_1932}). 

\lipsum[1] % dummy text fill

\begin{figure}
\centering
{\includegraphics[width=12cm,keepaspectratio]{example-image-a}}
\caption[Short caption goes here]{Long caption goes here}
\label{fig:introfigure1}
\end{figure}

\lipsum[1-5] % dummy text fill

\section{Information Box Example}

\lipsum[5] % dummy text fill

Some of the best supported and accepted models among biologists tend to be variation and selection models that indicate deleterious mutation accumulation in asexuals due to a lack of recombination (for example, M\"uller's ratchet; Box \ref{ibx:muller}; \cite{muller_relation_1964}), paired with the inability of asexuals to keep up with constantly changing selection pressures \parencite[for example the Red Queen hypothesis;][]{van_valen_new_1973}. 

\lipsum[3] % dummy text fill

\begin{infobox}[Example information box]
\lipsum[1] % dummy text fill
\label{ibx:muller}
\end{infobox}


\section{Table Example}

\lipsum[1] % dummy text fill

Several hypotheses have been proposed (\cref{tab:introtab1}) that draw from the major hypotheses explaining the prevalence of sexual reproduction (\cref{tab:introtab1}).

% Please add the following required packages to your document preamble:
% \usepackage{booktabs}
% \usepackage{graphicx}
\begin{table}[h]
\centering
\caption[Short table caption goes here]{Long table caption goes here}
\label{tab:introtab1}
\resizebox{\textwidth}{!}{%
\begin{tabular}{@{}llr@{}}
\toprule
\textbf{} & \textbf{Example Heading}                             & \textbf{Example Heading} \\ \midrule
          & \textbf{}                                       &                              \\
1         & Example Cell & Example Cell                   \\
          &                                                 & Example Cell               \\
          &                                                 & \multicolumn{1}{l}{}         \\
2         & Example Cell                         & Example Cell       \\
          &                                                 & Example Cell                \\
          &                                                 & \multicolumn{1}{l}{}         \\
3         & Example Cell                        & Example Cell                  \\
          &                                                 & Example Cell                   \\
          &                                                 & \multicolumn{1}{l}{}         \\
4         & Example Cell                          & Example Cell             \\
          &                                                 & Example Cell                   \\
          &                                                 & \multicolumn{1}{l}{}         \\
5         & Example Cell                     & Example Cell                   \\
          &                                                 & Example Cell         \\
          &                                                 & \multicolumn{1}{l}{}         \\
6         & Example Cell                             & Example Cell                   \\
          &                                                 & \multicolumn{1}{l}{}         \\
          &                                                 &                              \\ \bottomrule
\end{tabular}
}
\end{table}

\lipsum[1-5] % dummy text fill

\section{Footnotes example}

\lipsum[1] % dummy text fill

For example, many plants exhibit facultative apomixis\footnote[1]{\textsc{Apomixis:} \textit{Here considered as the production of offspring without syngamy, and genetic contribution from only one parent \parencite[for comparative terminology between plants and animals see][]{neiman_genetic_2014}}} to varying degrees, presumably in response to varying ecological parameters.

\lipsum[2] % dummy text fill

The pools (sometimes referred to as gnammas\footnote[2]{\textsc{Gnamma:} \textit{A word derived from the Nyungar language that refers to a rockhole capable of holding water}}) vary in shape, diameter and depth, and are classified depending on these dimensions.

\section{Fancy quote example}

\lipsum[1-3] % dummy text fill

\begin{fquote}Does habitat stability govern the relative success of sexual and parthenogenetic lineages? 
\end{fquote}

\section{References}
\sloppy % helps to align the text (less stringent threshold for gaps between words)
\printbibliography[heading=none]
\fussy % more stringent threshold for gaps between words

