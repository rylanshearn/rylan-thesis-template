
\fussy
\thispagestyle{empty}
\cleartooddpage

\noindent
The theory of evolution by natural selection \parencite{darwin_origin_1859} has had one of the greatest impacts on ourselves, and the world around us. We are closer to understanding the roots and circumstances under which our own species came into existence \parencite{stringer_genetic_1988}, while evolutionary theory has provided the basis for dealing with some of the biggest problems faced by humanity, whether it be understanding the dynamics of viral disease \parencite{faria_early_2014} or estimating the effects of climate change on the diversity of life \parencite{thuiller_consequences_2011}. More recently, the discovery of DNA \parencite{watson_molecular_1953}, then the development of sequencing technology \parencite[reviewed in][]{mardis_decade/s_2011}, has accelerated the fields of phylogenetics, genomics, and computational biology, marking the beginning of a new era of evolutionary biology \parencite{koboldt_next-generation_2013}. Despite having these tools at our disposal, there are still major gaps in our understanding of fundamental evolutionary theory. For instance the evolution of senescence, although having received much attention for over a century \parencite{weismann_essays_1889}, does not yet have a unified explanation \parencite{,monaghan_evolutionary_2008}. Another classic example is the long standing debate on what a unit of evolution (or species) is \parencite{de_queiroz_species_2007}. Perhaps the greatest remaining problem though, one that has been termed as "the Queen of problems in evolutionary biology" \parencite{bell_masterpiece_1982} is the reason why sexual reproduction is so prevalent throughout all eukaryota.

\section{The paradox of sex}
The importance of sexual reproduction has been contemplated since the beginnings of evolutionary theory \parencite{darwin_effects_1876}. One of the earliest theories suggested that sexual reproduction exists because it provides variation \parencite{weismann_significance_1889,weismann_amphimixis_1892}, which was later stated in the context of population genetics as providing accelerated evolution (\cref{fig:introfigure3}; \cite{,fisher_genetical_1930,muller_genetic_1932}). This theory was widely accepted until further study began to show that under this model, sexual reproduction is not advantageous if certain assumptions were violated, such as a large population size \parencite{crow_evolution_1965}, or mutations being unique events \parencite{maynard-smith_evolution_1968}. Soon after, this led to the real nature of the problem being articulated, the two-fold cost of males \parencite{maynard-smith_origin_1971}. 

\begin{figure}
\centering
{\includegraphics[width=12cm,keepaspectratio]{example-image-a}}
\caption[Basis of the Fisher-M\"uller accelerated evolution hypothesis]{Basis of the Fisher-M\"uller accelerated evolution hypothesis, why sexual populations may theoretically evolve more rapidly than asexuals. Mutations A, B and C are more advantageous than their alleles a, b and c, while a combination of all three (ABC) is most advantageous. In an asexual population, if mutations A, B and C occur in three separate lineages simultaneously, they cannot be combined between lineages, rather, they must occur successively in the same lineage over time. In sexual populations, the advantageous mutations can be combined between lineages with recombination, allowing them to increase in frequency independently of one another. After \textcite{maynard-smith_evolutionary_1989}}
\label{fig:introfigure3}
\end{figure}

The two-fold cost of males describes a hypothetical scenario where if two lineages are identical in every respect, except that one is sexual and the other asexual, the asexual lineage should produce twice as many female offspring as the sexual lineage \parencite{maynard-smith_origin_1971}. As it is only females that directly contribute to population growth rate, this model postulates the rapid out-competing of sexuals by asexuals (\cref{fig:introfigure1}). This model effectively displayed that the prevalence of sexual reproduction as opposed to asexual reproduction throughout taxa and time are paradoxical \parencite{williams_sex_1975}. This, alongside other disadvantages of sexual reproduction like the energy and risk associated with finding mates and the act of courtship, the slower speed of reproduction, and the risk associated with mixing genes with another individual \parencite{otto_resolving_2002}, should mean that sexual reproduction be an evolutionary dead end, when it is in fact the most common reproductive mode among eukaryotes \parencite{bell_masterpiece_1982}.

\begin{figure}
\centering
{\includegraphics[width=12cm,keepaspectratio]{example-image-a}}
\caption[The two-fold cost of sex]{The two-fold cost of sex. All else being equal than reproductive mode, asexual lineages should quickly outnumber sexual lineages, because males do not contribute toward population growth rate. After \textcite{maynard-smith_origin_1971}.}
\label{fig:introfigure1}
\end{figure}

Subsequently, a great deal of study has been devoted to attempting to explain advantages of sexual reproduction that may outweigh the severe theoretical costs. There are now a large number of hypotheses (\cref{tab:introtab1}), so many that a hypothesis classification scheme has been developed in order to better understand the underlying mechanisms contributing to the prevalence of sexual reproduction \parencite{kondrashov_classification_1993}. Several models have arisen that elaborate on \textcite{weismann_significance_1889,weismann_amphimixis_1892}, each giving different reasons for why the variation produced through recombination in sexual reproduction is beneficial. \textcite{kondrashov_classification_1993} termed these `variation and selection' models (\cref{tab:introtab1}). The `variation and selection' models can be further classified according to two factors; the first asks whether the cause of non-optimal genotypes comes from environmental change or deleterious mutations\footnote[1]{\textsc{Deleterious mutation:} \textit{A mutation that decreases the evolutionary fitness of an organism}}, and the second asks whether the cause happens by chance, or through selection \parencite{kondrashov_classification_1993}. The resulting four groups; Environmental stochastic, Environmental deterministic, Mutational stochastic, and Mutational deterministic represent unique combinations of the two factors.

% Please add the following required packages to your document preamble:
% \usepackage{booktabs}
% \usepackage{graphicx}
\begin{table}
\centering
\caption[Major theories explaining the prevalence of sexual reproduction]{Major theories explaining the prevalence of sexual reproduction, categorised according to the system proposed by \textcite{kondrashov_classification_1993}, where models based on variation and selection are categorised according to the cause of non-optimal or `bad' genotypes, and whether this occurs by chance, or due to selection. The earliest references eluding to each idea are given, regardless of whether they were officially formulated into hypotheses (for most hypotheses this can be found in \cite{kondrashov_classification_1993}).}
\label{tab:introtab1}
\resizebox{\textwidth}{!}{%
\begin{tabular}{@{}llr@{}}
\toprule
\textbf{}   & \textsc{Variation and Selection Models}                        & \textsc{Earliest references} \\
            & \textbf{}                                                      &                              \\
            & \textbf{Environmental stochastic}                              & \textbf{}                    \\
            & \textbf{(`Bad' genes from environmental change by chance)}     & \textbf{}                    \\
1           & Fisher-M\"uller hypothesis                                       & \cite{fisher_genetical_1930,muller_genetic_1932}     \\
2           & Beneficial --- Deletarious mutation interaction                & \cite{fisher_genetical_1930}                  \\
3           & Beneficial mutation --- Polymorphism interaction               & \cite{,manning_consequences_1983,manning_sex_1982}           \\
4           & Fluctuating selection                                          & \cite{seger_parasites_1988}      \\
5           & Tangled bank -- General purpose genotype                       & \cite{ghiselin_economy_1974}                \\
            &                                                                & \textbf{}                    \\
            & \textbf{Environmental deterministic}                           & \textbf{}                    \\
            & \textbf{(`Bad' genes from environmental change by selection)}  &                              \\
6           & Irreversible environment                                       & \cite{charlesworth_mutation_1993}            \\
7           & Fluctuating environment --- responsiveness (Red Queen)                    & \cite{van_valen_new_1973}           \\
8           & Fluctuating environment --- non-responsiveness                 & \cite{sturtevant_interrelations_1938}   \\
            &                                                                & \textbf{}                    \\
            & \textbf{Mutational stochastic}                                 & \textbf{}                    \\
            & \textbf{(`Bad' genes from deleterious mutations by chance)}    &                              \\
9           & M\"uller's ratchet                                               & \cite{muller_relation_1964}                  \\
            &                                                                & \textbf{}                    \\
            & \textbf{Mutational deterministic}                              & \textbf{}                    \\
            & \textbf{(`Bad' genes from deleterious mutations by selection)} &                              \\
10          & Kondrashov's Hatchet                                           & \cite{kondrashov_selection_1982}              \\
\textbf{}   &                                                                &                              \\ \midrule
\textbf{}   & \textsc{Immediate Benefit Models}                              & \textsc{Earliest references} \\
            &                                                                & \textbf{}                    \\
            & \textbf{Increased fitness of progeny}                          &                              \\
11          & Two parents benefit                                            & \cite{lloyd_benefits_1980}                   \\
12          & DNA repair by chromosome conjugation                           & \cite{farley_gametes_1982}                  \\
            &                                                                & \textbf{}                    \\
            & \textbf{Decreased deleterious mutations}                       &                              \\
13          & New mutations being deletions                                  & \cite{bengtsson_biased_1986}               \\
14          & New mutations being insertions                                 & \cite{ettinger_meiosis-selection_1986}                \\
15          & New epimutations being demethylations                          & \cite{holliday_possible_1988}                \\
            &                                                                & \textbf{}                    \\
            & \textbf{Increased selection efficiency}                        &                              \\
16          & Increased selection efficiency                                 & \cite{geodakyan_role_1965}                    \\
            &                                                                &                              \\ \midrule
            & \textsc{Structured Population Models}                          & \textsc{Earliest references} \\
            &                                                                &                              \\
17          & Parent offspring differences                                   & \cite{rice_parent-offspring_1983,rice_sexual_1983}                 \\
18          & Sib competition                                                & \cite{williams_sex_1975}               \\
19          & Increased competition with diversity                           & \cite{maynard-smith_evolution_1978}           \\
20          & Deterministically generated associations                       & \cite{serebrovsky_nekotoryye_1973}            \\
            &                                                                &                              \\ \bottomrule
\end{tabular}
}
\end{table}

Some models do not treat recombinational variation as being the mechanism for sexual reproduction being advantageous. Instead it is seen as an inadvertent consequence of some other process that is what \textcite{kondrashov_classification_1993} called the real `Immediate Benefit'. These were accordingly termed Immediate Benefit models (\cref{tab:introtab1}).

Several other models explain processes occurring in spatially structured populations. Although these hypotheses, like variation and selection models, can be classified as processes resulting from environmental change or deleterious mutations, stochastically or through selection, these are usually considered separately \parencite{maynard-smith_evolution_1978,kondrashov_classification_1993} as they account for differences between sub-populations, while variation and selection models only consider unstructured populations (\cref{tab:introtab1}).

Some of the best supported and accepted models among biologists tend to be variation and selection models that indicate deleterious mutation accumulation in asexuals due to a lack of recombination (for example, M\"uller's ratchet; Box \ref{ibx:muller}; \cite{muller_relation_1964}), paired with the inability of asexuals to keep up with constantly changing selection pressures \parencite[for example the Red Queen hypothesis;][]{van_valen_new_1973}. Unfortunately, the conditions that suit mathematical modelling tend to mean that complex processes that favour sex in the natural environment are overlooked, in other words, there are always unrealistic assumptions. Each model represents the compartmentalisation of fundamental processes within evolutionary biology (selection, mutation, recombination, migration and genetic drift\footnote[2]{\textsc{Genetic drift:} \textit{Changes in frequency of genotypes within a population due to the loss of particular genes by chance as individuals fail to reproduce}}), and it is becoming clear that a synthesis of these processes is needed \parencite{west_pluralist_1999,otto_resolving_2002}. Additionally, studies of real world examples measuring factors contributing toward recombination load\footnote[3]{\textsc{Recombination load:} \textit{Difference in fitness between offspring produced with, and without recombination}}, preferably using model organisms with both reproductive modes, or those that differ mainly in reproductive mode, are also needed to support theoretical syntheses \parencite{otto_resolving_2002}. Although the underlying mechanisms of recombination load are not well understood in the real world, their effects are clearly visible in a phenomenon that shows promise in revealing them; Geographic Parthenogenesis.

\begin{infobox}[M\"uller's ratchet]
M\"uller's ratchet illustrates a major issue for asexual lineages, that being how to remove deleterious mutations. In a finite population, individuals without deleterious mutations may be lost by genetic drift. In sexual organisms, the original genotype can be restored through sex and recombination between individuals that have varying genotypes. For asexual lineages without a mechanism for recombination however, the deleterious mutation is irreversible, and the whole lineage is left with it. Over time, this is repeated, and the genotype with one mutation is lost, leaving a whole population with two deleterious mutations. The process repeats through time, ultimately leading to the extinction of asexual lineages. M\"uller related this one-way, irreversible process to the mechanism of a ratchet. Each turn to the next notch represents the irreversible addition of a deleterious mutation \parencite{muller_relation_1964}.
\label{ibx:muller}
\end{infobox}


\section{Geographic Parthenogenesis}
\begin{sloppypar}
Geographic parthenogenesis, first observed by \textcite{vandel_parthenogen`ese_1928,vandel_parthenogen`ese_1940}, is a phenomenon whereby sexual and asexual lineages of the same or closely related species have differing geographic distributions. The phenomenon has since been documented by others, with asexuals tending to occur at higher latitudes (northern hemisphere), at higher altitudes, at the boundary of the species range, on islands rather than mainland habitats, in succession rather than climax communities, or in arid versus non-arid habitats \parencite{glesener_sexuality_1978,bell_masterpiece_1982,lynch_destabilizing_1984,kearney_why_2003,parker_jr_geographic_2002}. For example, in Europe there is a tendency for sexual lineages to occur in the Mediterranean area, while asexual lineages can occur everywhere \parencite{vandel_parthenogen`ese_1928}. On the other hand, in Australia there is a tendency for sexual lineages to occur in higher rainfall (or stable) areas, while asexuals occur in more arid (or unstable) habitats \parencite{kearney_why_2003}.
\end{sloppypar}

Cases like these have received much attention from evolutionary biologists, as the ecological properties of the different habitats that sexuals and asexuals occupy may uncover mechanisms of recombination load, and could thus give insight into the reason for the prevalence of sexual reproduction. Several hypotheses have been proposed (\cref{tab:introtab2}) that draw from the major hypotheses explaining the prevalence of sexual reproduction (\cref{tab:introtab1}). The hypotheses are not all mutually exclusive, but rather explain different mechanisms occurring at different scales of space and time.

% Please add the following required packages to your document preamble:
% \usepackage{booktabs}
% \usepackage{graphicx}
\begin{table}[h]
\centering
\caption[Major hypotheses explaining Geographic Parthenogenesis]{Major theories explaining Geographic Parthenogenesis, showing earliest known reference describing the hypothesis. Multiple references are given where hypotheses were published multiple times, or where the nomenclature for the same process has changed.}
\label{tab:introtab2}
\resizebox{\textwidth}{!}{%
\begin{tabular}{@{}llr@{}}
\toprule
\textbf{} & \textbf{Hypothesis}                             & \textbf{Earliest references} \\ \midrule
          & \textbf{}                                       &                              \\
1         & Superior colonisation \& reproductive assurance & \cite{baker_characteristics_1965}                   \\
          &                                                 & \cite{tomlinson_advantages_1966}               \\
          &                                                 & \multicolumn{1}{l}{}         \\
2         & Diverse sexual genotype                         & \cite{glesener_sexuality_1978}       \\
          &                                                 & \cite{hamilton_fluctuation_1981}                \\
          &                                                 & \multicolumn{1}{l}{}         \\
3         & General purpose genotype                        & \cite{vandel_parthenogen`ese_1928}                  \\
          &                                                 & \cite{baker_characteristics_1965}                   \\
          &                                                 & \multicolumn{1}{l}{}         \\
4         & Frozen niche variation                          & \cite{roughgarden_evolution_1972}             \\
          &                                                 & \cite{white_heterozygosity_1970}                   \\
          &                                                 & \multicolumn{1}{l}{}         \\
5         & Destabilising hybridisation                     & \cite{lynch_destabilizing_1984}                   \\
          &                                                 & \cite{paulissen_ecology_1988}         \\
          &                                                 & \multicolumn{1}{l}{}         \\
6         & Heterotic genotypes                             & \cite{white_heterozygosity_1970}                   \\
          &                                                 & \multicolumn{1}{l}{}         \\
7         & Hybrid intermediate genotypes                   & \cite{wright_weeds_1968}                   \\
          &                                                 & \cite{moore_evaluation_1977}         \\
          &                                                 & \multicolumn{1}{l}{}         \\
8         & General purpose genotype through polyploidy     & \cite{vandel_parthenogen`ese_1928}                   \\
          &                                                 & \multicolumn{1}{l}{}         \\
9         & Heterozygosity assurance                        & \cite{vrijenhoek_heterozygosity_1982}    \\
          &                                                 & \cite{beukeboom_evolutionary_1998} \\
          &                                                 & \cite{haag_new_2004}           \\
          &                                                 &                              \\ \bottomrule
\end{tabular}
}
\end{table}

A main advantage of being asexual is that only one individual is required to colonise a new habitat (Hypothesis 1, \cref{tab:introtab2}); sexuals on the other hand require a mate to simultaneously colonise \parencite{baker_characteristics_1965,tomlinson_advantages_1966}. This may explain short term successes of asexual lineages to colonise peripheral, unoccupied habitats. In central, species packed and highly competitive environments, the variation and selection models (\cref{tab:introtab1}) postulate that conditions would favour sex and recombination (Hypothesis 2, \cref{tab:introtab2}), because genetically diverse populations would have a selective advantage \parencite{glesener_sexuality_1978,hamilton_fluctuation_1981}.

The general purpose genotype hypothesis (Hypothesis 3, \cref{tab:introtab2}) describes a selective process that assumes parthenogenetic taxa are made up of multiple clones\footnote[4]{\textsc{Clone:} \textit{Here considered as a group of asexually reproducing organisms derived from an asexual or sexual ancestor, that are genetically identical to one another, while the plural `clones' is considered as multiple groups of these organisms that are genotypically distinct from one another}} that are constantly being generated from sexual ancestors \parencite{parker_jr_geographic_2002}. As clones are generated, each has a varying degree of environmental tolerance that, without recombination, is fixed in time. Over several generations, the clones that persist in the highest number of habitats, rather than the clones with the highest fitness in any given habitat, are selected. A similar model has also been proposed with variation not being derived from sexual ancestors, but from mutations within the clonal lineages themselves \parencite{lynch_destabilizing_1984}.

The frozen niche variation hypothesis (Hypothesis 4, \cref{tab:introtab2}), like the general purpose genotype hypothesis, also describes a process that assumes parthenogenetic taxa are composed of multiple clones being generated from a sexual ancestor \parencite{vrijenhoek_ecological_1984,vrijenhoek_factors_1979}. Clones are still generated with genotypes that are fixed in time, however the genotypes that tend to enable utilisation of resources that the sexual ancestor is dominating are outcompeted early. Other clones may have genotypes that enable utilisation of resources that the sexual ancestor is underutilising. In this case, there is the possibility of stable coexistence by partitioning of resources.

Another hypothesis called `Heterozygosity assurance' or the `Metapopulation hypothesis' (Hypothesis 9, \cref{tab:introtab2}) assumes that marginal populations are in fact parts of metapopulations with frequent subpopulation extinction and recolonisation (\cref{fig:introfigure4}). Recolonisation of marginal habitats will lead to genetic drift for both sexual and asexual lineages. The difference is that for sexuals, genetic drift will lead to inbreeding depression\footnote[5]{\textsc{Inbreeding depression:} \textit{Reduced fitness of an organism resulting from inbreeding and consequent increased homozygosity}} \parencite{charlesworth_inbreeding_1987}, whereas asexual lineages avoid inbreeding depression because without mating and segregation their homozygosity cannot be increased any further \parencite{haag_new_2004,vrijenhoek_heterozygosity_1982}.

\begin{figure}
\centering
{\includegraphics[width=12cm,keepaspectratio]{example-image-a}}
\caption[Heterozygosity assurance, and the distribution of sexual vs. asexual populations]{Heterozygosity assurance and the hypothetical distribution of sexual versus asexual populations in core versus marginal habitats. Genetic drift increases from core (bottom) to marginal (top) habitats due to decreasing population size (represented by circle size), isolation, and frequent extinction-recolonisation events. Asexuals are hypothesised to do better in marginal habitats because they do not experience inbreeding depression as a result of drift like sexuals do. After \textcite{haag_new_2004}.}
\label{fig:introfigure4}
\end{figure}

Hybridisation is also likely to play a role in geographic parthenogenesis. This has been viewed as a mechanism by which a general purpose genotype can rapidly evolve. This is largely because many viable hybrid lineages are polyploid. This can give increased gene combinations enabling adaptation to a broader range of ecological niches (Hypothesis 8, \cref{tab:introtab2}; \cite{lynch_destabilizing_1984,vandel_parthenogen`ese_1928}). Evidence for this hypothesis can be found in polyploid asexual anostracans \parencite{maniatsi_is_2011}, weevils \parencite{stenberg_evolution_2003} and ostracods \parencite{adolfsson_evaluation_2009} from Europe which have wider distributions than diploid asexuals. Alternatively, deleterious effects of hybridisation between parthenogens and their sexual ancestors may lead to differing distributions of reproductive mode, a process known as `destabilising hybridisation' (Hypothesis 5, \cref{tab:introtab2}; \cite{lynch_destabilizing_1984}). 

Hybrids have also been seen as preserving intermediate genotypes (Hypothesis 7, \cref{tab:introtab2}). If the hybrid genotype is inferior to the two sexual ancestors, it must occupy marginal niches \parencite{wright_weeds_1968,moore_evaluation_1977}. Alternatively, the hybrid lineage may have a heterotic genotype \parencite{white_heterozygosity_1970}, in which case it could be more generalist and occupy a wider distribution (Hypothesis 6, \cref{tab:introtab2}). As polyploids are often the result of hybridisation \parencite[for example][]{,kearney_why_2003,kearney_waves_2006,schon_slow_1998,kearney_waves_2006}, these processes may be acting simultaneously, and careful interpretation is required to disentangle the contribution of ploidy level, and heterosis toward the observed distribution.

\begin{sloppypar}
So although there is evidence of differential geographic distributions of reproductive mode, there are multiple genetic, population and metapopulation level processes that may be contributing toward observed trends. Additional to these compounding processes is the fact that when comparing environmental parameters to reproductive mode, there can be a mismatch of variable types that could be obscuring true trends in the natural environment. For instance, there may be a distinct boundary between the geographic distributions of sexual and asexual representatives of a hypothetical organism. We are then forced to characterise each region by a multitude of environmental parameters in order to explain the differing distributions \parencite[for example;][]{,adolfsson_evaluation_2009}. It is argued here that the lumping of numerical data into categories in this way could be removing resolution in any potential ecological trends. 

Ideally, treating reproductive mode as a numerical variable can have potential for the relative contribution of each environmental parameter on recombination load to be better quantified. For example, many plants exhibit facultative apomixis\footnote[6]{\textsc{Apomixis:} \textit{Here considered as the production of offspring without syngamy, and genetic contribution from only one parent \parencite[for comparative terminology between plants and animals see][]{neiman_genetic_2014}}} to varying degrees, presumably in response to varying ecological parameters \parencite{eckert_clonal_1993,dorken_severely_2001,cosendai_geographical_2013}. For mobile taxa however, the degree of asexuality is more complicated to quantify, as in spatially large populations with patchy distributions there is always the possibility that rare males \parencite[for example][]{chaplin_sex_1994} are not being sampled, which results in observed sex ratios that are not representative of the population when subsampling very large habitats. Ideally, the study of small, insular habitats that can be sampled more rigorously could be used as models for the same processes occurring in larger ecological systems. In the next section, we will explore an aquatic model system that is yielding insight into many evolutionary and ecological processes for these reasons \parencite{brendonck_pools_2010}; the rock outcrop pools system.
\end{sloppypar}

\section{The rock outcrop pools experimental system} \label{intro:rockpools}
Many rock outcrops or inselbergs throughout the world contain pools at their summit \parencite{brendonck_pools_2010} that have weathered into the rock over time through thermal expansion, pressure release, frost riving, crystal growth \parencite{bayly_aquatic_2011} or accelerated chemical weathering through water collecting at the rock surface \parencite{lister_microgeomorphology_1973}. The pools (sometimes referred to as gnammas\footnote[7]{\textsc{Gnamma:} \textit{A word derived from the Nyungar language that refers to a rockhole capable of holding water}}) vary in shape, diameter and depth, and are classified depending on these dimensions \parencite{twidale_gnammas_1963}. Pools at the summit tend to be shallow, pan or bucket shaped, and highly temporal, being completely dependant on rainfall for recharge \parencite{,brendonck_pools_2010}. Consequently, the aquatic taxa that inhabit these pools have developed strategies to deal with periods of desiccation, for example the production of drought resistant eggs \parencite{vanschoenwinkel_relative_2008}.

Rock outcrops often hold a number of pools at their surface that are very well defined (\cref{fig:introfigure5}a) and proximal to one another, compared to the distance between rock outcrops holding pools, which can be tens to hundreds of kilometers. The simplicity and abundance of temporary pools on multiple widely distributed rock outcrop summits (\cref{fig:introfigure5}b) make them ideal for comparative analyses at local, regional and inter-continental scales \parencite{blaustein_why_2001}, and this is likely to have been the main driver of their recent popularity as models for understanding larger ecosystem dynamics \parencite{brendonck_regional_2000,vanschoenwinkel_relative_2008,vanschoenwinkel_role_2007,jocque_faunistics_2006,pellowe-wagstaff_ecology_2014,pinceel_fairy_2013}. Additionally, they are often regularly distributed along ecological gradients \parencite{,vanschoenwinkel_role_2007} facilitating the quantification of effects of environmental parameters on metacommunity or metapopulation processes. 

\begin{figure}
\centering
  \subfloat[Multiple pools on top of a rock outcrop]{\includegraphics[width=12cm,keepaspectratio]{example-image-a}}
  \newline
  \subfloat[Multiple rock outcrops as islands in the regional landscape matrix]{\includegraphics[width=12cm,keepaspectratio]{example-image-b}}
  \newline
\caption[A representation of the spatial organisation of rock outcrop pools]{A representation of the spatial organisation of rock outcrop pools, both within (a), and between (b) rock outcrops. Redrawn after \textcite{brendonck_pools_2010}.}
\label{fig:introfigure5}
\end{figure}

While there are other natural aquatic systems like tree holes and pitcher plants that are ideal for modelling larger ecosystem processes \parencite{ellis_evaluating_2006,ng_hierarchical_2009,pandit_contrasts_2009}, they do not support the same diversity of taxa as that found in rock outcrop pools \parencite{jocque_faunistics_2006} which includes some crustaceans that are known to reproduce both sexually and asexually. For example, many species of non-marine Ostracoda are commonly found in rock outcrop pools \parencite{vanschoenwinkel_community_2009,pinder_granite_2000,bayly_invertebrate_1982,bayly_aquatic_2011}, and these species often have a mixture of parthenogenetic females, sexual females, and sexual males, that can all exist sympatrically \parencite{martens_reproductive_1998}. Usually, parthenogenetic females and sexual females of the same species are morphologically indistinguishable. However, sex ratios can be indicative of reproductive mode, whereby any ratio close to 50:50 is likely to be sexual, while populations with a mixture of both modes are likely to have female-skewed ratios \parencite{,martens_reproductive_1998}. It is argued here that sex ratio could be considered as an approximation of recombination load, with relatively even sex ratios indicating a higher recombination load on asexual lineages of the same species. Usually such measures are restricted to laboratory based experiments, for as discussed above, one cannot be certain whether rare males have been missed when sampling large metapopulations with patchy distributions. For the rock outcrop pool system however, populations exist within small, manageable, well defined habitats that are easily sampled \parencite{,brendonck_pools_2010}, which could potentially give more accurate estimates of sex ratio. Additionally, rock outcrop pools are highly similar habitats, with the difference between pools mainly limited to climatic factors \parencite{vanschoenwinkel_role_2007}. Thus, the study of these pools throughout important climatic and ecological gradients enables the limitation of habitat variability to a smaller number of parameters than what is usually possible in ecosystem based research \parencite{vanschoenwinkel_role_2007}, providing the opportunity to relate a small number of ecological parameters to reliable numerical sex ratios as response variables. This combination of factors is suggested here, in aid of the need for real world evidence of mechanisms leading to recombination load \parencite{otto_resolving_2002}.

\begin{sloppypar}
In such a study system, one must have a thorough understanding of whether all females counted really belong to a common lineage, to be sure that parthenogenetic females of two different species are not mistakenly counted together. Thus, a thorough systematic understanding is essential, and methods of species delimitation that are appropriate to this study system must be applied. In the next section, the idea of the species concept will be explored, as well as species concepts that are appropriate for organisms that reproduce both sexually and asexually.
\end{sloppypar}

\section{On `species' and reproductive mode} \label{intro:species}
Species are widely adopted as the major unit of evolution \parencite{,mayr_growth_1982}, still there have been many decades of debate on what a species is, with a large number of species concepts developed. Some concepts are conflicting \parencite{mayden_hierarchy_1997}, where the application of different concepts will result in different species boundaries, or different numbers of species being estimated. In recent years, there has been a transition away from confusing species conceptualisation with species delimitation \parencite{,de_queiroz_species_2007}, and a new approach suggests that many species concepts represent properties that diverging lineages acquire successively through time. During divergence, the acquisition of traits that support `species concepts' (isolation, monophyly etc) will be gradual, and \textcite{,de_queiroz_species_2007} argues that this is the time where many species concepts come into conflict (\cref{fig:introfigure6}). To overcome this, species can be considered as separately evolving metapopulation lineages, while `secondary' species concepts can be considered as evidence in support of, or documenting the degree of lineage divergence. The following is a consideration of how each major class of these `lines of evidence' might apply to organisms and study systems that involve both sexual and asexual modes of reproduction, whether these lineages can be considered a single species, and the implications this may have on findings made.

\begin{figure}
\centering
{\includegraphics[width=12cm,keepaspectratio]{example-image-a}}
\caption[The relationship between species divergence and secondary species concepts]{Depiction of the relationship between species divergence and secondary species concepts. Several secondary species concepts form criterion (labelled 1 to 9) that may be attained by diverging lineages through time. Before any criterion are observed, or after all criterion are observed in two lineages, there is unanimous agreement amongst criterion or `secondary species concepts' on the number of species present (species A and B). However, within the `grey zone', the criterion are in conflict because each represents varying thresholds, depending on the organism in question. Redrawn after \textcite{de_queiroz_species_2007}.}
\label{fig:introfigure6}
\end{figure}

\begin{sloppypar}
Perhaps the most deceivingly intuitive species concept is the biological species concept, which requires that individuals within the hypothesised species interbreed and produce fertile offspring \parencite{huxley_statistical_1940,mayr_systematics_1942,dobzhansky_mendelian_1950}. However, for parthenogenetic lineages that do not need to interbreed with other individuals, this type of concept cannot be employed. Furthermore, many parthenogenetic lineages result from interspecific hybridisation \parencite{schon_slow_1998,kearney_waves_2006,kearney_why_2003}, so any concepts of reproductive isolation may not even apply to some sexually reproducing populations. The limitations of the biological species concept are extensive \parencite[reviewed in][]{mayden_hierarchy_1997} and this has only been applicable in sexually reproducing organisms under conditions where reproduction can be readily observed in the environment, or in controlled laboratory conditions.
\end{sloppypar}

The Evolutionary species concept is broadly applicable in many organisms, and considers species as lineages that maintain their identity from others throughout space and time, having their own evolutionary fate and history \parencite{,simpson_principles_1961,wiley_evolutionary_1978,wiley_evolutionary_2000}. The broad applicability of this concept raises questions of what an identity is, and what is considered a separate evolutionary history. For example, a clonal lineage arising from a sexual population being considered a different species will depend on whether it is able to then hybridise with the sexual ancestor. If one were to study this system on a brief time scale, the clonal lineage may be considered a different species, while on a long time scale, intermittent clonal production and hybridisation \parencite{lynch_destabilizing_1984} may be considered as a component of intra-population dynamics. As the Evolutionary species concept relies so heavily on the time scale being considered, it should be used with caution in studies that involve only single sampling events, and organisms with complicated patterns of reproduction.

In contrast to the Evolutionary species concept, Ecological species concepts \parencite{van_valen_ecological_1976,andersson_driving_1990} can consider sexual, asexual, and hybrid lineages as separate species, so long as they have distinct ecological roles. This concept can just as easily consider sexual and asexual lineages as the same species, so long as they also share that ecological role \parencite{van_valen_ecological_1976}. This fits with our current understanding of geographic parthenogenesis, where sexual and asexual lineages of the same or closely related species have nested ecological boundaries that represent an equal ecological role, but varying breadth of tolerance \parencite[for example][]{adolfsson_evaluation_2009}. Obviously however, there are severe limitations to delineating species based on this concept alone, as a large number of ecological parameters would need to be quantified in order to delineate two species competing over similar niches.

The cohesion species concept \parencite{templeton_meaning_1989,templeton_species_1998} was in fact proposed after a critique of the lack of consideration for variation in reproductive mode within the Evolutionary and biological species concepts, and was an attempt at rectifying this situation. A series of genetic and demographic exchangeability mechanisms given in \textcite{templeton_meaning_1989} enable the evaluation of the proposed species groups based on the degree of cohesion rather than isolation. However, it has been argued that this was more of a guideline for identifying species \parencite{,mayden_hierarchy_1997} rather than a primary concept that has potential for explaining all species level diversity.

Similarly, the phenetic species concept \parencite{,michener_diverse_1970,sokal_biological_1970,sneath_numerical_1973,sneath_phenetic_1976}, a synonym of the Morphological species concept, is also based on methodology, with species being defined as having a higher degree of observable similarity within, than between species. This concept is basically a classification system used mainly by taxonomists, and does not necessarily treat species as lineages. For this reason it has not been accepted as a candidate for explaining all species level diversity \parencite{,mayden_hierarchy_1997}. However, such systems are fundamental in determining whether sexual and parthenogenetic lineages should belong within the same species, for example based on morphological similarity of females.

With the rise of phylogenetic systematics, several concepts have also been developed that deal with differentiation that occurs below the species level. These phylogenetic concepts aim to define species as the smallest possible group of organisms that are diagnosable \parencite{,nelson_systematics_1981,cracraft_species_1983,nixon_amplification_1990}, monophyletic \parencite{,rosen_fishes_1979,donoghue_critique_1985,mishler_morphological_1985} or both. Whether diagnosable, or monophyletic, all phylogenetic concepts have the ability to recognise sexual and asexual organisms as species \parencite{,mayden_hierarchy_1997}. Moreover, since reproductive mode is seen as an attribute, rather than determining the species identity, there is the possibility of viewing sexual and parthenogenetic lineages as the same species so long as they form the smallest possible groups of organisms that are diagnosable and/or monophyletic. Thus it is argued here that these concepts should be preferred in systems where different reproductive modes are possible, and where appropriate (DNA sequence) data are available.

Several species delineation methods have been developed to be applied genetic data that draw from phylogenetic species concepts, including the Generalised Mixed Yule Coalescent method \parencite{pons_sequence-based_2006} or the use of genetic variability thresholds \parencite{,lefebure_relationship_2006}. The $K:\theta$ rule (Box \ref{ibx:Ktheta}) also draws from the evolutionary species concept, and the phenetic species concept, in that species are considered as independently evolving populations for the time span in question, and there must be a higher degree of similarity within than between species \parencite{,birky_using_2010}. The method is now widely used to delineate sexual species \parencite{,birky_species_2013}, asexual species \parencite{,birky_using_2010}, and species that exhibit both modes of reproduction \parencite{,martens_/textitbennelongia_2013,martens_nine_2012,shearn_review_2012}.

\begin{infobox}[The $K:\theta$ Method]
The basis of the $K:\theta$ method is that random extinction of lineages or metapopulation structuring can produce phylogenetic gaps between clades within a species \parencite{birky_using_2010}. Within species, individuals should have descended from a common ancestor $2N_e$ generations ago ($N_e$ being effective population size), and gaps separating clades within species should be ephemeral in nature. During speciation however, the gap between two previously intraspecific clusters deepens to far beyond $2N_e$. The $K:\theta$ method identifies species clades as those clusters that are separated by a time to their most recent shared common ancestor of $4N_e$ generations, as this represents the upper 95 \% confidence limit that the gap between clusters has not been formed as a result of random genetic drift.

The method uses mean sequence divergence ($K$) between candidate species clades as an estimator for the time to their most recent common ancestor, and $N_e$ $\mu$ is estimated with $\theta = \pi /(1-4\pi/3)$, where $\mu$ is the mutation rate per base pair, per generation, and $\pi$ is the mean sequence difference between individuals within one proposed species clade. Thus two clades are considered not a result of random genetic drift, but complete speciation at the 95 \% confidence limit when the mean sequence divergence between two proposed clades exceeds four times the estimate of within clade nucleotide diversity for both clades (i.e. when $K\geq4\theta$).

\label{ibx:Ktheta}
\end{infobox}

Through this brief exploration of the main classes of species concepts, the secondary concepts, or `lines of evidence' as advocated by \textcite{,de_queiroz_species_2007} that are likely to be relevant to studies of variation in reproductive mode in temporary pool organisms are likely to be; 1) Phylogenetic, enabling the observation and quantification of divergence between lineages, using objective criteria like the $K:\theta$ method \parencite{,birky_using_2010} and most importantly to show whether all females of a given population exhibit a high degree of genetic similarity; and 2) particularly where genetic data may be unavailable, classification based concepts like the phenetic species concept, although not founded on notions of lineage divergence \parencite{michener_diverse_1970,sokal_biological_1970}, can be suitable surrogates for understanding phenotypic similarity between sexual and asexual females. So long as particular attention is paid toward morphological characters of the female gender in sexual species descriptions as well asexual ones, the phenetic and phylogenetic concepts or `lines of evidence' under the separately evolving metapopulation lineage concept \parencite{,de_queiroz_species_2007} will ensure that differential distributional patterns in reproductive mode are in fact occurring within the same, or closely related species, rather than being two distributions of highly diverged lineages (in which case, different distributions may be driven by factors other than recombination load). With this understanding of the taxonomic framework required for studies of differential distribution in reproductive mode, suitable candidate organisms can now be chosen for use as models for the study of variation in reproductive mode throughout temporary pools, and throughout the rock outcrop pools experimental system (\cref{intro:rockpools}).

\section{The model organism}
As mentioned briefly in \cref{intro:rockpools}, many non-marine ostracods use both sexual and parthenogenetic modes of reproduction, often having both reproductive modes within a single species (sometimes referred to as `mixed reproduction', for more details see \cref{morph}). At least two lines of evidence contribute toward being able to consider asexually and sexually reproducing lineages of ostracods as either the same or different species. The first is that sexual and parthenogenetic lineages can show genetic and morphological coherence, which has been the basis for considering them as the same species based on phenetic and phylogenetic based species concepts \parencite{martens_reproductive_1998}. Secondly, intermittent hybridisation between parthenogenetic females and sexual males can be considered as part of the internal dynamics of the same evolving metapopulation \parencite{martens_reproductive_1998}, and thus a single species according to \textcite{,de_queiroz_species_2007}. Alternatively, if parthenogenetic lineages are indeed isolated from hybridisation over time, they could be considered 'ancient asexual' lineages \parencite{schon_slow_1998}.

Patterns of global scale geographic parthenogenesis have been observed within the non-marine ostracod genus \textit{Ilyodromus} \cite{sars_contributions_1894} (\cref{fig:introfigure7}) with sexual representatives observed exclusively in temporary pools and wetlands of Australia, and parthenogenetic populations occurring in many other regions of the world \parencite{,mckenzie_palaeozoogeography_1971}. Although this was only based on very few records, they also suggest that patterns may exist within Australia. Furthermore, the genus also has representatives in rock outcrop pools of Western Australia \parencite{,pinder_granite_2000,bayly_invertebrate_1982} that are distributed along an ecological gradient primarily associated with aridity. The presence of an organism with potentially varied reproductive mode throughout rock outcrop pools on an ecological gradient provides opportunities for major hypotheses on causes of recombination load in natural systems to be evaluated in light of trends in other animals and plants in Australia. On this continent, parthenogenesis is thought to occur more in arid zones where limited and unpredictable rainfall is thought to leave empty but unstable habitats \parencite{,kearney_waves_2006,kearney_why_2003}. Meanwhile, high, predictable rainfall and consequent species packed habitats at the continental margin \parencite{glesener_sexuality_1978,hamilton_fluctuation_1981} are thought to give sex and recombination a selective advantage though accelerated evolution \parencite{fisher_genetical_1930,muller_genetic_1932}. These existing patterns in Australia pose the main question behind this thesis:

\begin{fquote}Does habitat stability govern the relative success of sexual and parthenogenetic lineages? 
\end{fquote}

This would be expected because of three major hypotheses associated with geographic parthenogenesis:

\begin{description}
  \item[Superior colonisation and reproductive assurance] \hfill \\
  Theoretically superior colonisation ability and reproductive assurance \parencite{,baker_characteristics_1965,tomlinson_advantages_1966} should give parthenogenetic lineages a selective advantage in marginal (less stable) habitats.
  \item[Heterozygosity assurance] \hfill \\
  Sexual lineages should have an inferior ability to (re)colonise marginal (less stable) habitats, as is suggested by the theoretical inbreeding depression experienced by colonising sexual lineages \parencite{,vrijenhoek_heterozygosity_1982,beukeboom_evolutionary_1998,haag_new_2004}.
  \item[Diverse sexual genotype] \hfill \\
  Central (more stable) habitats should favour sexual populations for their higher genetic variation, as is suggested by the diverse sexual genotype hypothesis \parencite{,glesener_sexuality_1978,hamilton_fluctuation_1981}.
\end{description}

\begin{figure}
\centering
{\includegraphics[width=12cm,keepaspectratio]{example-image-a}}
\caption[Impression of a typical \textit{Ilyodromus} ostracod]{Illustration of a typical \textit{Ilyodromus} ostracod (not to scale)}
\label{fig:introfigure7}
\end{figure}

Although \textit{Ilyodromus} shows much promise for use as a model for understanding this question, there are several factors preventing its immediate adoption. The systematic knowledge of nominal species remains poor, with many species having been described in low detail over a century ago \parencite{sars_contributions_1894,sars_freshwater_1896,king_australian_1855}. Additionally, there has often been a focus on male diagnostic characters in taxonomic descriptions rather than female characters, which although very useful for fast identification in ecological studies and biodiversity estimation, it is not suitable in cases where one must be certain of whether females of a sexual and an asexual lineage may be the same species (see arguments made in \cref{intro:species}). Furthermore, considering that Australia has been flagged with high potential for undescribed non-marine ostracod diversity \parencite{martens_global_2008}, and that rock outcrop pools are foci for diversification \parencite{pinder_granite_2000}, there are likely to be several undescribed species of \textit{Ilyodromus} encountered within the rock outcrop study system. This, paired with the poor understanding of nominal species necessitates a revision of nominal species in higher detail, to determine whether species encountered are already described, or new to science. The systematic placement of several species as belonging within this genus has also been questioned \parencite{martens_taxonomy_2001}. Several are reported to exhibit a high degree of similarity to taxa of another subfamily (Isocypridinae) in at least three publications \parencite{martens_taxonomy_2001,de_deckker_ostracoda_1981,mckenzie_freshwater_1966}. Thus, as well as revising nominal species there is a need for revision of the boundaries between \textit{Ilyodromus} and other closely related genera. 

Through the presentation of a series of papers that have either been published, or prepared for submission to peer reviewed journals at the time of writing, this thesis aims to revise the systematics of \textit{Ilyodromus}, then, using this genus as a model organism within the rock outcrop pools system, endeavouring to quantify the relative importance of drivers behind differential distributions of reproductive mode. This is achieved through a series of sub-aims that seek to:

\begin{enumerate}
\item
redescribe extant nominal species of \textit{Ilyodromus} from the Australian region, with particular attention to diagnostic female characters.
\item
revise the identity of the type species of \textit{Ilyodromus}, and thus characters that are considered typical of the genus.
\item
investigate the reports of high morphological similarity between some species of \textit{Ilyodromus} and genera of Isocypridinae
\item
investigate and describe any undescribed species level diversity of \textit{Ilyodromus} within the rock outcrop pools system of south-western Australia, with particular attention to diagnostic female characters.
\item
determine any patterns of differential distribution of reproductive mode for \textit{Ilyodromus} populations occurring in the rock outcrop pools system along an ecological gradient in south-western Australia.
\item
determine the strongest environmental drivers behind any patterns of differential distribution of reproductive mode for \textit{Ilyodromus} populations occurring in the rock outcrop pools system along an ecological gradient in south-western Australia.
\end{enumerate}

\section{References}
\sloppy
\printbibliography[heading=none]
\fussy

